\section{Summarization Steps}

\begin{figure}[H]
\centering
\begin{tikzpicture}[node distance=0.7cm, auto]
\node (input) [] {Input(s)};
\node (interpretation) [block, right =of input] {Interpretation};
\node (transformation) [block, right =of interpretation] {Transformation};
\node (generation) [block, right =of transformation] {Generation};
\node (output) [right =of generation] {Output};
\draw [->] (input) -- (interpretation);
\draw [->] (interpretation) -- (transformation);
\draw [->] (transformation) -- (generation);
\draw [->] (generation) -- (output);
\end{tikzpicture}
\caption{\cite{lloret_text_2008} Summarization Steps}
\label{fig:summarization_steps}
\end{figure}

\begin{enumerate}
\item \textit{\textbf{Interpretation:}} The very first step for text summarization is that of \textit{syntactic parsing}, i.e. generating a \textit{parse tree}. Depending on which parser works best for us, we will either use \textbf{CoreNLP} or \textbf{spaCy}. From this, we need to obtain an ASP representation so that we can analyze it further.
\item \textit{\textbf{Transformation:}}
  \begin{itemize}
  \item For the membership problem (see problem statement in Chapter \ref{chapter:introduction}), using either representation of the inputs, we need to establish similarity between the information contained in the two inputs. This would involve a combination of some of the techniques mentioned in Chapter \ref{chapter:literature_review}. For instance, we could try and build matching \textit{lexical chains} using ASG.
  \item For the other two problems, our program would simply be the outcome of iterative fine-tuning, making use of ILP tasks in ASG. In this mindset, our approach would be symbolic, although we might want to implement a few of the advanced techniques as part of this fine-tuning. In the end, what we want is a condensed version of the original document's ASP representation.
  \end{itemize}
\item \textit{\textbf{Generation:}}
  \begin{itemize}
  \item For the two other problems, the output is a boolean value saying whether or not the text and summary match. To generate this value we need to have an algorithm that decides on the answer, based on the analysis from the previous step.
  \item Otherwise, we need to take the transformed text representation, and turn it into grammatically-correct English using ASGs.
  \end{itemize}
\end{enumerate}

\section{Datasets}

For our purposes, we have found a dataset of short children's stories \cite{noauthor_examples_nodate}\cite{noauthor_short_nodate}, whose title can help us get a clue as to what the summary should contain. A few of these stories are under a page in length, do not have any quotations, and use simple vocabulary and sentence structures, thus would be suitable for this project.

More suitable examples though would come from English comprehension exercises for KS1 students, as the examples are even simpler. To this end, we can extract stories from reading comprehension worksheets available on \textbf{\href{https://www.k5learning.com/reading-comprehension-worksheets/first-grade-1}{K5Learning}}, as well as from English reading test materials on \textbf{\href{https://www.gov.uk/government/publications/key-stage-1-tests-2019-english-reading-test-materials}{GOV.UK}} and \textbf{\href{https://mathsmadeeasy.co.uk/ks1-revision/key-stage-1-sat-english-exam-tests/}{MathsMadeEasy}}.

In addition, we can also create our own short texts in order to get started, but also to check edge cases when the project is farther along. For illustrative purposes, below is a succinct example passage with a possible summary for it:

\begin{displayquote}
One day, a little boy named Peter Little was curious. He was interested in stars and planets. So he was serious in school and always did his homework. When he was older he studied physics and maths. He studied hard for his exams and became an astrophysicist. Now he is famous.
\end{displayquote}

\begin{displayquote}
\textbf{Summary:} \textit{Peter Little was interested in space so he studied hard and became a famous astrophysicist.}
\end{displayquote}

We will also want to have a bank of synonyms for the interpretation step, but this may also be useful when it comes to generating a English summary that is well written. One such database providing true ontology data is \textbf{\href{http://www.conceptnet.io}{ConceptNet}}, which provides an API. This particular service also gives additional information such word types and synonyms, which may prove particularly useful when trying to understand the concepts behind a text as well as when negation is used.

For additional knowledge, it might also be helpful to use \textbf{\href{https://wiki.dbpedia.org/develop/datasets}{DBpedia}}, a service which extracts and compiles data from Wikipedia via an API. A good use case for this would be when we are faced with a proper noun and would like to know how it links semantically to the rest of the text.

\section{Milestones}

\begin{table}[H]
\centering
\begin{tabular}{@{}ll@{}}
\toprule
\textbf{Milestone 1} & Parse tree generation \\ \midrule
\textbf{Milestone 2} & Representation of parse tree in ASG \\ \midrule
\textbf{Milestone 3} & Learn semantic constraints in ASG to recognize correct summaries \\ \midrule
\textbf{Milestone 4} & Transcription into English summary  \\ \bottomrule
\end{tabular}
\caption{Project milestones}
\label{table:milestones}
\end{table}

\begin{table}[H]
\centering
\begin{tabular}{@{}cl@{}}
\toprule
\textbf{Expected Completion} & \multicolumn{1}{c}{\textbf{Task}} \\ \midrule
Mid January & Go through tiny example manually \\
Late January & Submit report and find good datasets \\
Mid March & Create working system (some steps may still be manual) \\
April & Integrate all the steps so it can be run automatically \\ \bottomrule
\end{tabular}
\caption{Project timeline}
\label{table:timeline}
\end{table}

Table \ref{table:timeline} shows the timeline for this project, along with a set of major milestones listed in Table \ref{table:milestones}.