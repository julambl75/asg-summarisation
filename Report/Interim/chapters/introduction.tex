\label{chapter:introduction}

\section{General Problem}

In general, the task of summarization in NLP (natural language processing) is to produce a shortened text which covers the main points expressed in a longer text (given as input). In order to do this, a system performing such a task must analyze/process the input to be able to extract from it the most important information.

\section{Specific Problem}

There are three different overlapping problems we may wish to pursue with the use of answer set programming (ASG):

\begin{enumerate}
\item Given a short text of about a page long, for example a short story aimed at young children, to provide a summary in multiple sentences. The goal here would be to identify which sentences in the passage are important, extract only these, and then possibly apply some post-processing to link them a bit better.
\item Given a very short text less than a page long comprised of very brief sentences, for example reading comprehension exercises for Key Stage 1 students, to provide a summary of just one or two sentences in length. What is important here is to establish which \textit{chunks} (i.e. subparts of sentences)  are important, thereby semantically learning the main descriptions and actions occurring in the text. From this, we can construct a meaningful \textit{abstract} (see Chapter \ref{chapter:background}).
\item Given a short text of about a page long (as in the first problem) as well as a summary of a few sentences, to establish whether the latter is a summary of the passage. The goal here would be to understand the semantics of both inputs, use a metric to determine how well they match, and then apply a decision criterion to give a boolean answer.
\end{enumerate}

\section{Challenges}

In all of the problems cited above, we can expect to encounter the below challenges.

\begin{challenge}[Proper Nouns]
When our program encounters a proper noun, it must be able to relate it to other words in the text, by knowing what type of word it is and also having a bit of background knowledge about it. For example, in the text ``He went to Paris yesterday. After arriving in France, he travelled across the city to his hotel.", our system would need to relate the words ``Paris", ``France" and ``city", but keep ``hotel" as a distinct entity.
\end{challenge}

\begin{challenge}[Periphrasis]
At the very least, our solution must be able to link synonyms, such as ``house" and ``home", as well as antonyms, such as ``happy" and ``sad". On a more advanced note, it should also have enough semantic analysis capabilities to know that ``To the left of the goat is a pig." and ``To the right of the pig is a goat." mean the same thing.
\end{challenge}

\begin{challenge}[Negation]
We should ensure that our program knows how to interpret sentences which involve negation, which can be tricky for a semantic analysis tool. On a basic level we should be able to relate a negated word with its antonym, for example ``not happy" with ``sad" or ``unhappy". In addition, we can try as a stretch goal to achieve a much more complete understanding of the semantics of negation in English, allowing the solution to know that ``Nobody came to the picnic." and ``Everyone stayed home." might be equivalent in some contexts.
\end{challenge}

\begin{challenge}[Event Chronology]
Understanding the order of events in a story extremely important, as getting this wrong could make an entire summary incorrect. In order to achieve a good understanding of this, our program must be able to interpret connectives as well as time markers. Good examples for the former would be ``Before eating he washed his hands." and ``He washed his hands and then ate". In the latter case, if a text contains the string ``Today is Monday. In two days she has an exam.", our solution needs to be context-aware and know that the exam is held on a Wednesday.
\end{challenge}

\begin{challenge}[Sentence Subject]
Aside from our program being able to identify the subject of a basic sentence, it should also be able to understand which sentences have the same subject, so that it can link actions and descriptions with subjects. In passage ``John likes music. He has a guitar", our solution should be able to pick up on the fact that John has a guitar.
\end{challenge}

\begin{challenge}[Adjectives And Adverbs]
When a sentence contains adjectives and/or adverbs, the program should be able to know what subjects or actions they apply to. The following is an extremely difficult but all-encompassing example: ``The white cat quickly jumped over the brown squirrel who was slowly eating a fresh cobnut".
\end{challenge}

\begin{challenge}[Context]
A solution must be highly context-aware, and use the correct definition of a word depending on the context. For example, the word ``address" can either be a verb meaning ``to speak to", or the location of a physical place.
\end{challenge}