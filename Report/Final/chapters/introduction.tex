\label{chapter:introduction}

\section{General Problem}

In general, the task of summarization in NLP (natural language processing) is to produce a shortened text which covers the main points expressed in a longer text (given as input). In order to do this, a system performing such a task must analyze/process the input to be able to extract from it the most important information.

\section{Specific Problem}

There are three different overlapping problems we may wish to pursue with the use of answer set programming (ASG):

\begin{enumerate}
\item Given a short text of about a page long, for example a short story aimed at young children, to provide a summary in multiple sentences. The goal here would be to identify which sentences in the passage are important, extract only these, and then possibly apply some post-processing to link them a bit better.
\item Given a very short text less than a page long comprised of very brief sentences, for example reading comprehension exercises for Key Stage 1 students, to provide a summary of just one or two sentences in length. What is important here is to establish which \textit{chunks} (i.e. subparts of sentences)  are important, thereby semantically learning the main descriptions and actions occurring in the text. From this, we can construct a meaningful \textit{abstract} (see Chapter \ref{chapter:background}).
\item Given a short text of about a page long (as in the first problem) as well as a summary of a few sentences, to establish whether the latter is a summary of the passage. The goal here would be to understand the semantics of both inputs, use a metric to determine how well they match, and then apply a decision criterion to give a boolean answer.
\end{enumerate}

\section{Objectives}

\textcolor{red}{\textbf{\hl{TODO}}}