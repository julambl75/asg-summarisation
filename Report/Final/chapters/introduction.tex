\label{chapter:introduction}

In general, the task of summarization in NLP (natural language processing) is to produce a shortened text which covers the main points expressed in a longer text (given as input). In order to do this, a system performing such a task must analyse/process the input to be able to extract from it the most important information.

\section{Motivations}

In recent years, state-of-the-art systems that accomplish text summarization have relied largely on Machine Learning. These include Bayesian classifiers, hidden Markov models, neural networks and fuzzy logic, among others \cite{kiyani_survey_2017}. Given a training corpus, along with some careful preprocessing as well as fine-tuning of hyper-parameters and feature extraction functions, such systems are able to produce effective summaries. 

However to learn what is a summary, these systems require a very large amount of data, and take a long time to train. In contrast, using logic means that we can hard-code this definition directly into our program, avoiding the problem of randomness and uncertainty. By carefully constructing its structure, we can get results with just a short list of rules, and know that it will always produce a complete and valid output with respect to the background knowledge we encode into it.

\section{Objectives}

The main goal of this project is to explore the task of text summarization via logic-based learning with Answer Set Grammars (ASG). Below you will find the principal objectives which were established as being vital to achieving this goal.

\begin{objective}[TODO]
TODO
\end{objective}

Given a brief paragraph of text, for example a short story aimed at young children, we should be able to provide a summary in multiple sentences. The goal here is to extract the most important information from all the relevant sentences, and from this generate one or more grammatically-correct summaries.

Although the bulk of the summarization logic should revolve around ASGs, pre- and/or post-processing may reveal itself beneficial in obtaining better-quality results.

%In almost all these papers, the first step is to generate a parse tree in order to identify the different parts of speech and get a better understanding of a text. Once a parse tree is obtained, the task becomes to find out which details are important, and which can be omitted from the summary. Finally, the summary must be grammatically correct and provide enough information such that it serves a purpose to the reader.
%
%To know what is important...

\textcolor{red}{\textbf{\hl{TODO fix}}}

\section{Approach Overview}

In what follows, we shall give a brief overview of the various tasks that were undertaken and completed as part of the project. Where relevant, references to the corresponding sections are provided.

The approach described in this paper can be diagrammatically represented as a three step pipeline, as seen in Figure \ref{fig:main_pipeline}. Although the focus of this project is a mechanism written in ASG, it relies on a number of Python scripts to coordinate the flow of information.

\begin{figure}[H]
\centering
\begin{tikzpicture}[node distance=0.3cm, auto]
\node (story) [] {Story};
\node (preprocessor) [block, right =of story] {Preprocessor};
\node (asg) [block, right =of preprocessor] {SumASG};
\node (score) [block, right =of asg] {Post-Processing/Scoring};
\node (summaries) [right =of score] {Scored Summaries};
\draw [->] (story) -- (preprocessor);
\draw [->] (preprocessor) -- (asg);
\draw [->] (asg) -- (score);
\draw [->] (score) -- (summaries);
\end{tikzpicture}
\caption{Main Pipeline}
\label{fig:main_pipeline}
\end{figure}

We begin by describing the essential role of the \textsc{Preprocessor} in (Chapter \ref{chapter:preprocessor}). Given an input story, its goal is to simplify the story's sentences into a simpler and more consistent structure that will then be easier to parse by ASG (Section \ref{sec:tokenization_scoring}).

%The central part of the pipeline, which performs story summarization, revolves around the ASG task of \textsc{SumASG}. To build on top of this foundation, we also have the \textsc{Preprocessor}, as well as some final post-processing and scoring mechanisms, all written in Python. We will call this augmented task \textsc{SumASG*}, and describe each step in the following chapters. A diagram of the entire pipeline can be seen in 


\subsection{Example}

Throughout this paper, we shall be using the example of the story of Peter Little to illustrate the different steps of our pipeline, as shown below in Figure \ref{fig:peter_little}.

\begin{figure}[H]\
\begin{subfigure}{\textwidth}
\begin{displayquote}
There was a curious little boy named Peter Little. He was interested in stars and planets. So he was serious in school and always did his homework. When he was older, he studied mathematics and quantum physics. He studied hard for his exams and became an astrophysicist. Now he is famous.
\end{displayquote}
\caption{Story of Peter Little}
\vspace{\baselineskip}
\end{subfigure}
\begin{subfigure}{\textwidth}
\begin{displayquote}
\textbf{A.} Peter Little was interested in space so he studied hard and became a famous astrophysicist.\\
\textbf{B.} Peter Little was curious about astronomy. He was always serious in school, and now he is famous.
\caption{\textit{Reference summaries}}
\end{displayquote}
\end{subfigure}
\caption{Example of the task of summarization for the story of Peter Little}
\label{fig:peter_little}
\end{figure}

\section{Contributions}

The main contribution of this project to the field of NLP is the creation of an end-to-end logic-based system capable of text summarization, without the need of any training whatsoever, as would be the case with a typical Machine Learning-based approach these days.

More specifically, ...

\textcolor{red}{\textbf{\hl{TODO more in-depth}}}

\newpage

\section{Objectives}

In all of the problem cited above, should keep the below objectives in mind.

\begin{objective}[Proper Nouns]
When our program encounters a proper noun, it must be able to relate it to other words in the text, by knowing what type of word it is and also having a bit of background knowledge about it. For example, in the text ``He went to Paris yesterday. He arrived in France, then he travelled across the city to his hotel.", our system would need to relate the words ``Paris", ``France" and ``city", but keep ``hotel" as a distinct entity.
\end{objective}

\begin{objective}[Periphrasis]
At the very least, our solution must be able to link synonyms, such as ``house" and ``home". On a more advanced note, it might hopefully also have enough semantic analysis capabilities to know that ``To the left of the goat is a pig." and ``To the right of the pig is a goat." mean the same thing.
\end{objective}

\begin{objective}[Negation]
We should try to make our program able to interpret sentences which involve negation, which can be tricky for a semantic analysis tool. On a basic level we should be able to relate a negated word with its antonym, for example ``not happy" with ``sad" or ``unhappy".
\end{objective}

\begin{objective}[Event Chronology]
Understanding the order of events in a story extremely important, as getting this wrong could make an entire summary incorrect. In order to achieve a good understanding of this, our program must be able to interpret connectives as well as time markers. A good example for the former would be ``He washed his hands and then ate". As a stretch objective, if a text contains the string ``Today is Monday. In two days she has an exam.", our solution should be context-aware and know that the exam is held on a Wednesday.
\end{objective}

\begin{objective}[Sentence Subject]
Aside from our program being able to identify the subject of a basic sentence, it should also be able to understand which sentences have the same subject, so that it can link actions and descriptions with subjects. In passage ``John likes music. He has a guitar", our solution should be able to pick up on the fact that John has a guitar.
\end{objective}

\begin{objective}[Adjectives And Adverbs]
When a sentence contains adjectives and/or adverbs, the program should be able to know what subjects or actions they apply to. The following is an extremely difficult but all-encompassing example: ``The white cat quickly jumped over the brown squirrel who was slowly eating a fresh cobnut".
\end{objective}

\begin{objective}[Context]
A solution must be highly context-aware, and use the correct definition of a word depending on the context. For example, the word ``address" can either be a verb meaning ``to speak to", or the location of a physical place.
\end{objective}