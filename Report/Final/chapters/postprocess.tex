\section{Overview}

Once we have obtained potential sentences from ASG to be used in a summary, we can now post-process these as explained in Section \ref{sec:summary_creation}. By combining them in different ways, we are able to form summaries. From these, we will retain the highest scoring ones, according to the metric detailed in Section \ref{sec:scoring}. A diagram illustrating these steps is shown below in Figure \ref{fig:postprocess_pipeline}.

\begin{figure}[H]
\centering
\begin{tikzpicture}[node distance=0.55cm, auto]
\node (summary_sentence_1) [block] {Summary Sentence 1};
\node (summary_sentence_2) [block, below =of summary_sentence_1] {Summary Sentence 2};
\node (summary_sentence_3) [below =of summary_sentence_2] {...};
\node (summary_sentence_4) [block, below =of summary_sentence_3] {Summary Sentence n};
\node (post_process_1) [block, right =of summary_sentence_1] {Post-Process};
\node (post_process_2) [block, below =of post_process_1] {Post-Process};
\node (post_process_3) [below =of post_process_2] {...};
\node (post_process_4) [block, below =of post_process_3] {Post-Process};
\node (combine) [block, right =of post_process_2] {Combine};
\node (summary_1) [block, above right =of combine] {Summary 1};
\node (summary_2) [block, below =of summary_1] {Summary 2};
\node (summary_3) [below =of summary_2] {...};
\node (summary_4) [block, below =of summary_3] {Summary m};
\node (score_1) [right =of summary_1] {Score};
\node (score_2) [below =of score_1, right =of summary_2] {Score};
\node (score_3) [below =of score_2] {...};
\node (score_4) [below =of score_3, right =of summary_4] {Score};
\draw [->] (summary_sentence_1) -- (post_process_1);
\draw [->] (summary_sentence_2) -- (post_process_2);
\draw [->] (summary_sentence_4) -- (post_process_4);
\draw [->] (post_process_1) -- (combine);
\draw [->] (post_process_2) -- (combine);
\draw [->] (post_process_4) -- (combine);
\draw [->] (combine) -- (summary_1);
\draw [->] (combine) -- (summary_2);
\draw [->] (combine) -- (summary_4);
\draw [->] (summary_1) -- (score_1);
\draw [->] (summary_2) -- (score_2);
\draw [->] (summary_4) -- (score_4);
\end{tikzpicture}
\caption{Post-Processing / Scoring Steps}
\label{fig:postprocess_pipeline}
\end{figure}

\section{Summary Creation}
\label{sec:summary_creation}

The output of \textsc{SumASG} is a list of sentences, each of which could potentially appear in the final summary.

\subsection{Post-Processing}

\subsection{Combining}

\section{Scoring}
\label{sec:scoring}

\section{Expandability}

\textcolor{red}{\textbf{\hl{TODO}}}