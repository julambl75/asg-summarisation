\label{chapter:evaluation}

\section{General Idea}

As the vast majority of modern text summarization frameworks are based on machine learning, it makes sense to compare the performance \textsc{SumASG*} with that of a neural network.

More specifically, we should generate a set of stories which we can give to our framework in order to obtain corresponding summaries. We can then use this as training data for an encoder-decoder, to see if it is able to learn how \textsc{SumASG*} creates summaries.

If the neural network is able to learn to generate similar summaries, then we can consider our framework to be sane.

\section{Datasets}

In order to generate the required number of stories, we have used words from \href{http://www.wordfrequency.info/}{wordfrequency.info}. This database contains 5,000 individual English words, of which 1,001 are verbs, 2,542 nouns and 839 adjectives.

\section{Libraries}

For this task, we have chosen to use a library called \textbf{\href{http://web.archive.org/web/20190516161631/https://www.clips.uantwerpen.be/pages/pattern-en}{Pattern}}, which allows us to conjugate verbs, as well as toggle nouns between singular and plural.

We also take advantage of the \textbf{\href{https://www.datamuse.com/api/}{Datamuse API}}, which lets us find words which are semantically related to a given word in a certain way.

\section{Story Generation}

For each story we chose a noun from our dataset, which we shall refer to as the \textit{topic}. We then construct four sentences which revolve around this \textit{topic}.

\subsection{Sentence Generation}

We will begin by detailing how each sentence is generated, starting with a few necessary definitions. Throughout this section, it is important to keep in mind that the goal here is to create a story that is as lexically and semantically coherent as possible, which is tricky to do algorithmically.

\subsubsection{Definitions}

\begin{definition}[Hyponym]
A \textit{hyponym} is a word with more specific meaning than another word; ``computer" is a \textit{hyponym} of ``machine".
\end{definition}

\begin{definition}[Hypernym]
A \textit{hypernym} is a semantic superclass of a word; ``vehicle" is a \textit{hypernym} of ``bus".
\end{definition}

\begin{definition}[Holonym]
A \textit{holonym} of something is one of its constituents; ``lightbulb" is a \textit{holonym} of ``lamp".
\end{definition}

\begin{definition}[Meronym]
A \textit{meronym} is an object which something is part of; ``house" is a \textit{meronym} of ``kitchen".
\end{definition}

\subsubsection{Lexical Common Nouns}

Along with the story's \textit{topic}, we also generate a set of \textit{lexical common nouns}. If the \textbf{Datamuse API} is able to find \textit{synonyms} of our \textit{topic} which also belong to our dataset of nouns, then these become the story's \textit{lexical common nouns}.

In addition, we query from the \textbf{Datamuse API} for verbs that are related to the chosen \textit{topic}. This becomes our set of \textit{lexical verbs}. If it is empty, then we make it the singleton set containing the verb ``to be".

Since we don't know how general or specific this randomly selected \textit{topic} is, we may not find any. In this case, we try the same procedure for \textit{hypernyms} and finally \textit{hyponyms}. If we still are unable to find any (which is very rare), then we pick a new random \textit{topic}.

\subsubsection{Subject}

For the \textit{subject} of a sentence, we draw a noun from our \textit{lexical common nouns}. If this word is singular, then we need a determiner, which can be either ``the" or ``a". We also ask the \textbf{Datamuse API} to find us an adjective which is often modified by the chosen \textit{subject} noun, and is part of our dataset of words. If none is found, then we do not need to use an adjective.

\subsubsection{Verb}

We chose a verb at random from our set of \textit{lexical verbs}, conjugating it in the past tense so that it agrees with the sentence's \textit{subject}.

\subsubsection{Object}

For the \textit{object} of our sentence, we look at the \textit{subject} and \textit{verb}. Using the \textbf{Datamuse API} we try and find a word...

\subsubsection{Example}

\subsection{Action Creation}

\section{Training}

\section{Results}

\textcolor{red}{\textbf{\hl{TODO}}}