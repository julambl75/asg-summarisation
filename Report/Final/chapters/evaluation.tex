\label{chapter:evaluation}

\section{General Idea}

As the vast majority of modern text summarization frameworks are based on machine learning, it makes sense to evaluate \textsc{SumASG*} against a neural network.

More specifically, we should generate a set of stories which we can give to our framework in order to obtain corresponding summaries. We can then use this as training data for an \textit{encoder-decoder}, to see if it is able to learn how \textsc{SumASG*} creates summaries.

If the neural network is able to learn how to generate these summaries, then we can consider our framework and the summarization rules it uses to be sane.

\section{Story Generation}

To give a more generalized representation of what \textsc{SumASG*} can do, we shall create two different types of stories: those with \textit{conjunctive summaries}, and those with \textit{descriptive summaries}.

In this section, we go more into depth about these two types of randomly-generated stories, and discuss how they are created.

\subsection{Libraries}

For this task, we have chosen to use a library called \textbf{\href{http://web.archive.org/web/20190516161631/https://www.clips.uantwerpen.be/pages/pattern-en}{Pattern}}, which allows us to conjugate verbs, as well as toggle nouns between singular and plural.

We also take advantage of the \textbf{\href{https://www.datamuse.com/api/}{Datamuse API}}, which lets us find words which are semantically related to a given word in a certain way.

\subsection{Datasets}

In order to generate the required number of stories, we have used words from \href{http://www.wordfrequency.info/}{wordfrequency.info}. This database contains 5,000 individual English words, of which 1,001 are verbs, 2,542 nouns and 839 adjectives.

For each story we chose a noun from our dataset, which we shall refer to as the \textit{topic}. We will later construct sentences that revolve around this \textit{topic}.

In addition, we query from the \textbf{Datamuse API} for what we call a \textit{lexical verb}, i.e. one that is related to the story's \textit{topic}. If one cannot be found, then we default to the verb ``to be".

\subsection{Main Sentence Generation}

We will begin by detailing how what we call \textit{main sentences} are generated, starting with a few necessary definitions. Throughout this section, it is important to keep in mind that the goal here is to create a story that is as lexically and semantically coherent as possible, which is tricky to do algorithmically.

It is important to note that all \textit{main sentences} for the same story share a common \textit{subject} and \textit{verb}.

\subsubsection{Definitions}

\begin{definition}[Holonym]
A \textit{holonym} of something is one of its constituents; ``lightbulb" is a \textit{holonym} of ``lamp".
\end{definition}

\begin{definition}[Meronym]
A \textit{meronym} is an object which something is part of; ``house" is a \textit{meronym} of ``kitchen".
\end{definition}

\subsubsection{Subject}

For the \textit{subject} of a \textit{main sentence}, we use the story's \textit{topic}. This being a singular noun, we need to add a determiner, which can be ``the" or ``a".

We also ask the \textbf{Datamuse API} to find us an adjective which is often modified by the chosen \textit{subject} noun, and is part of our dataset of words. If none are found, then we do not need to use an adjective.

\subsubsection{Verb}

Here we use the \textit{lexical verb}, conjugating it in the past tense so that it agrees with the sentence's \textit{subject}.

\subsubsection{Object}

For the \textit{object} of our sentence, we look at the story's \textit{topic} and \textit{lexical verb}. Using the \textbf{Datamuse API} we try and find a noun which often appears right after this verb, and which is related to all of the nouns we have used thus far in the story. With 50\% probability we ask it to be a \textit{holonym} of the \textit{topic}, otherwise it should be a \textit{meronym}.

In the same way as we did for the \textit{subject}, we try and find an adjective often modified by the chosen noun. Sometimes it will be the case that no noun was found, but it is possible in English to have an adjective as the only word in the \textit{object}.

The determiner is added as for the \textit{subject}; if there is no noun we do not use one.

\subsubsection{Example}

We take the example of generating a sentence for a story whose \textit{topic} is ``football". In this case, we have two choices for our \textit{lexical verb}: ``to match" and ``to pitch".

For the \textit{subject}, we use the \textit{topic} ``football"; an adjective commonly used to modify it is ``professional". Here, we can use the determiner ``a".

We can then conjugate the verb ``to pitch" in the past tense, which becomes ``pitched".

For the \textit{object}, we take into account our \textit{topic} ``football" to find a \textit{holonym} of this word which often appears after the \textit{verb} ``pitched". In this case the \textbf{Datamuse API} returns the word ``reception", resulting in the adjective ``warm" being chosen to accompany it.

As a result, we end up with the following \textit{main sentence}:

\begin{displayquote}
A professional football pitched the warm reception.
\end{displayquote}

\textcolor{red}{\textbf{\hl{TODO}}}

\subsection{Conjunctive Summaries}

Stories of this type consist of three \textit{main sentences}. Because of the way in which we have implemented the summarization rules in \textsc{SumASG}, one of the sentences in the corresponding summary should be a combination of two of two of the input story's. That is to say, its \textit{object} should consist of these two sentences' \textit{objects}, joined together using the conjunction ``and". An example is shown in Figure \ref{fig:conjunctive_summary_example}.

\begin{figure}[H]\
\begin{subfigure}{\textwidth}
\begin{displayquote}
The publication printed a lightning. The publication printed a movement. The publication printed the stereotype.
\end{displayquote}
\caption{Story}
\vspace{\baselineskip}
\end{subfigure}
\begin{subfigure}{\textwidth}
\begin{displayquote}
The publication printed a lightning and a movement. The publication printed the stereotype.
\end{displayquote}
\caption{Summary}
\end{subfigure}
\caption{Example of a story with a \textit{conjunctive summary}}
\label{fig:conjunctive_summary_example}
\end{figure}

\textcolor{red}{\textbf{\hl{TODO underline important parts in example?}}}

\subsection{Descriptive Summaries}

In contrast, this type of story consists of a single \textit{main sentence}, one which does contain an adjective in the \textit{object} position.

We use the second of its two sentences to expand on the first. To be more precise, the \textit{object} of this sentence is the same as in the first, apart from the fact that we assign an adjective as is done for typical \textit{main sentences}. However the \textit{subject} here is the preposition ``it", while the \textit{verb} is the verb ``to be" conjugated in the past tense.

The idea for a \textit{descriptive summary} is that it will be identical to the first sentence of its corresponding story, but augmented with the adjective coming from the second sentence, as illustrated in Figure \ref{fig:descriptive_summary_example}.

\begin{figure}[H]\
\begin{subfigure}{\textwidth}
\begin{displayquote}
The heavy traffic transported the birthday. It was the isolated birthday.
\end{displayquote}
\caption{Story}
\vspace{\baselineskip}
\end{subfigure}
\begin{subfigure}{\textwidth}
\begin{displayquote}
The heavy traffic transported the isolated birthday.
\end{displayquote}
\caption{Summary}
\end{subfigure}
\caption{Example of a story with a \textit{descriptive summary}}
\label{fig:descriptive_summary_example}
\end{figure}

\subsection{Action Creation}

Using a Python script, we generate the corresponding \textit{actions} as would \textsc{SumASG\textsubscript{1}}, creating the necessary additional leaf nodes for our general grammar in ASG. We do not use \textsc{SumASG\textsubscript{1}} to do this mainly for performance reasons, but also as it is not necessary. Because of the way in which we have created our stories, \textit{simplification} would not change the sentence structure whatsoever.

\subsection{Summary Generation}

For each story, we feed the generated \textit{actions} and leaf nodes directly into \textsc{SumASG\textsubscript{2}}, skipping the first half of the \textsc{SumASG*} pipeline. After \textit{scoring}, we pick an entry at random from the \textit{top summaries}.

\section{Neural Network}

\subsection{Datasets}

Using the mechanism described above, we generate a number of story/summary pairs: 3591 to be used for training, 399 for validation and 10 for testing.

\subsection{Tools}

To allow for greater flexibility, we have chosen to use a highly versatile open-source framework called \textbf{\href{https://github.com/OpenNMT/OpenNMT-py}{OpenNMT-py}} for training our neural network.

In addition, we preprocess the data using Stanford's \textbf{\href{https://nlp.stanford.edu/projects/glove/}{GloVe}} pre-trained word embeddings, giving our network a head-start when it comes to semantics.

\subsection{Encoder-Decoder Architecture}

Our \textit{encoder} and \textit{decoder} share embeddings for a vocabulary of size 4239, its contents being internally represented using a vector of size 500. They both use a two-layer LSTM with \textit{dropout} of 0.25 and \textit{hidden size} of 500. Additionally, our \textit{decoder} uses \textit{global attention}.

\textcolor{red}{\textbf{\hl{TODO}}}

\subsection{Training}

The neural network was trained using an Adam optimizer with a \textit{learning rate} of 0.001 and \textit{batch size} of 50. In order to preserve the GloVe word embeddings across training, we fix them at the start and use them for both the \textit{encoder} and \textit{decoder}. Training was done over a period of 10,000 steps (i.e., 200 epochs), validating every 10 epochs. It took 39 epochs for training accuracy to reach 99\%, while validation accuracy slowly increased over time to reach about 53\%.

\textcolor{red}{\textbf{\hl{TODO}}}

\subsection{Results}

At first glance, it may seem as though this \textit{encoder-decoder} was overfitting the training data. However, it is important to keep in mind that multiple valid summaries may exist for a given input story, and that the target summary was not necessarily the same as the one generated by the network.

Unfortunately, due to framework constraints, it is impossible to provide the training script with multiple target summaries. However, it is simple to run the trained network on our test stories, and then compare the results to the many summaries generated by \textsc{SumASG*}, as seen in Figure \ref{fig:neural_network_testing}.

\begin{figure}[H]
\begin{subfigure}{\textwidth}
\begin{displayquote}
\textbf{1.}\\
\textbf{2.}\\
\textbf{3.}\\
\textbf{4.}\\
\end{displayquote}
\caption{Test stories}
\end{subfigure}
\begin{subfigure}{\textwidth}
\vspace{\baselineskip}
\begin{displayquote}
\textbf{1.}\\
\textbf{2.}\\
\textbf{3.}\\
\textbf{4.}\\
\end{displayquote}
\caption{Summaries generated by the \textit{encoder-decoder}}
\end{subfigure}
\begin{subfigure}{\textwidth}
\vspace{\baselineskip}
\centering
\begin{tabular}{@{}lllll@{}}
\toprule
BLEU score                                        & 1 & 2 & 3 & 4 \\ \midrule
Predicted summary                                 & 0 & 0 & 0 & 0 \\
Closest \textsc{SumASG*} summary & 0 & 0 & 0 & 0 \\ \bottomrule
\end{tabular}
\caption{Maximum BLEU scores against original story}
\end{subfigure}
\caption{Evaluation results}
\label{fig:neural_network_testing}
\end{figure}

\textcolor{red}{\textbf{\hl{TODO}}}

\section{Takeaways}

To sum up what we have learned from this experiment, we can use a table to outline the main differences between our approach and using a neural network, as shown in Figure \ref{fig:takeaways}.

\begin{table}[H]
\centering
\begin{tabular}{@{}L{0.25\textwidth}L{0.4\textwidth}L{0.35\textwidth}@{}}
\toprule
                                        & Neural network                                                                                              & \textsc{SumASG*}                        \\ \midrule
Rules                                   & Learnable using state-of-the-art \textit{encoder-decoders}                                 & Written directly into program  \\
Training required                       & Yes; can take a long time                                                                                   & No                             \\
Examples required                       & Vast amounts for training                                                                                   & None                           \\
Expansion                               & Need to retrain                                                                                             & Can be used directly           \\
Coherence of result                     & Extremely tied to nature and diversity of training corpus                                                   & Similar on all parsable texts  \\
Output \textit{tokens} & Can be irrelevant or \texttt{<unk>} \textit{token} & Always taken from input text   \\
Termination                             & Always produces output, regardless of whether input is valid English                                        & Sometimes returns no summaries \\ \bottomrule
\end{tabular}
\caption{Main differences between \textsc{SumASG*} and neural networks used for the task of text summarization}
\label{fig:takeaways}
\end{table}