\label{chapter:asg}

\section{Overview}

Our use ASG is two-fold. Firstly, we pass in each sentence from the story to ASG to obtain its semantic representation in ASP. Secondly, we take these actions and use ASG rules to generate possible summary components. These will later be post-processed and turned into actual valid summaries. A diagram of the two ASG steps is shown below in Figure \ref{fig:asg_pipeline}.

\begin{figure}[H]
\centering
\begin{tikzpicture}[node distance=0.55cm, auto]
\node (sentence_1) [block] {Sentence 1};
\node (sentence_2) [block, below =of sentence_1] {Sentence 2};
\node (sentence_3) [below =of sentence_2] {...};
\node (sentence_4) [block, below =of sentence_3] {Sentence n};
\node (learn_action_1) [block, right =of sentence_1] {Learn Action};
\node (learn_action_2) [block, right =of sentence_2, below =of learn_action_1] {Learn Action};
\node (learn_action_3) [right =of sentence_3, below =of learn_action_2] {...};
\node (learn_action_4) [block, right =of sentence_4, below =of learn_action_3] {Learn Action};
\node (gen_summaries) [block, right =of learn_action_2] {Generate Summaries};
\node (summary_sentence_1) [block, above right =of gen_summaries] {Summary Sentence 1};
\node (summary_sentence_2) [block, right =of gen_summaries, below =of summary_sentence_1] {Summary Sentence 2};
\node (summary_sentence_3) [right =of gen_summaries, below =of summary_sentence_2] {...};
\node (summary_sentence_4) [block, right =of gen_summaries, below =of summary_sentence_3] {Summary Sentence m};
\draw [->] (sentence_1) -- (learn_action_1);
\draw [->] (sentence_2) -- (learn_action_2);
\draw [->] (sentence_4) -- (learn_action_4);
\draw [->] (learn_action_1) -- (gen_summaries);
\draw [->] (learn_action_2) -- (gen_summaries);
\draw [->] (learn_action_4) -- (gen_summaries);
\draw [->] (gen_summaries) -- (summary_sentence_1);
\draw [->] (gen_summaries) -- (summary_sentence_2);
\draw [->] (gen_summaries) -- (summary_sentence_4);
\end{tikzpicture}
\caption{ASG Steps}
\label{fig:asg_pipeline}
\end{figure}

\section{Internal Representation}

In order to model the structure of English, we have created a CFG that has a similar hierarchy to that of a parse tree in NLP.

\section{Learning Actions}

We firstly need to convert the preprocessed story's sentences from English into our 

\subsection{Formalization}

\section{Generating Summary Sentences}

\section{Expandability}

\textcolor{red}{\textbf{\hl{TODO}}}