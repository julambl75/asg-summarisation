\label{chapter:improvements}

Had there been more time to finalise the project, there are a number of possible improvements we could have implemented. In what follows, we discuss immediate next steps which could be taken to improve each part of the pipeline.

\section{Preprocessor}

In its current state, \textsc{SumASG} expects positive sentences only, and the only form of punctuation recognized is the full stop.

\subsection{Negation}

In order to support negation, we would need to modify the structure of \textsc{SumASG}'s internal representation (see Chapter \ref{chapter:asg}). However, to achieve a better semantic understanding in \textsc{SumASG*}, we could add some more simplification logic to the \textsc{Preprocessor}.

After having implemented this, the phrase ``not happy" would be transformed into the word ``sad".

\subsection{Lists}

At the moment, \textsc{SumASG} can parse a list of length 2 at the most, i.e. a conjunction of two items. By adding a transformation to the \textsc{Preprocessor} before we modify the punctuation (see Subsection \ref{subsec:punctuation}), we could overcome this limitation. Intuitively, this would mean going from a sentence with a an $n$-item list, to $\floor{\frac{n}{2}}$ sentences with two objects and one sentence with a single object (if $n$ is odd).

For instance, the sentence ``Bob had a book, a computer and a chair." would be split into ``Bob had a book and a computer. Bob had a chair.".

\section{ASG}

Throughout the development of \textsc{SumASG}, it was a constant struggle to try and find the right balance between expressibility, summary pertinence and computational efficiency.

\subsection{Missing English Structures}

Although our general grammar allows for a wide range in terms of the words that can be used to form a sentence, to no extent does it cover even a tenth of the sentences that are used in formal or informal English. Even if you were to consider only sentences consisting of exactly one clause, \textsc{SumASG} is incapable of understanding most non-general structures commonly used in English.

By greatly simplifying the input story using the \textsc{Preprocessor}, we were able to alleviate a large part of this struggle. However if we were to use our general grammar for a task other than summarization, we would most likely run into issues due to loss of information.

\subsection{Missing Summarization Rules}

In its current implementation, \textsc{SumASG\textsubscript{2}} only uses 12 summary generation rules, one of which simply repeats the given \textit{action}. While this is largely sufficient to demonstrate the potential of our approach, in no way can it be used as is in a production text summarization tool.

In order to augment this suite of rules, it would be simple to define a \textit{mode bias} to learn summary generation rules. With just a single example of a story and its corresponding summary, ASG could generate multiple such rules, allowing us to build up a large collection of these. Unfortunately, this is infeasible due to performance reasons.

\subsection{Speed}

Apart from readability, the main reason for trying to keep our general grammar's structure simple, and the number of summarization rules restricted, has to do with computational cost.

The more complexity we allow in terms of expressible sentences, the more expensive it is to use our general grammar.

Similarly, the more summarization rules we create, the longer it takes to generate summaries. On top of this, having more potential \textit{summary sentences} means that we end up with more summaries to score, many of which could be syntactically different but semantically equivalent.

In order to increase complexity without a detrimental impact on performance, we would need to either optimize ASG itself to run faster with our framework, or use more powerful machines.

\section{Post-Processing / Scoring}

Although our approach to post-processing and \textit{scoring} works well for the simple stories we have been using, it remains limited in terms of scope.

\subsection{Grammatical Shortcomings}

First of all, we do not revert all the simplification changes made by the \textsc{Preprocessor}. This can lead to a linguistically poor summary, where the same word or name is repeated multiple times, rather than using synonyms or personal pronouns.

Worse than this, we can end up with sentences that would never be written by a human. Because the \textsc{Preprocessor} moves all adverbs to the end of the sentence in which they appear, and is quite eager to homogenize synonyms, summaries generated by \textsc{SumASG*} may end up ``sounding wrong".

\subsection{Better Summary Selection}

Another issue is that we can easily end up with a very large list of summaries. Because the mechanism used to score them is not very advanced, it cannot determine for sure that one particular summary is better than all the others. Instead, we usually end up with multiple entries that all have the same maximum score.

We would therefore need to build much more intelligence into this system if we wanted the program to always return a single summary, one that is humans would also consider optimal.