\section{Overview}

In order to prepare the story to be parsed and summarized by \textsc{SumASG}, we have created the \textsc{Preprocessor}. As we will loose information in summary anyway, it is completely acceptable to... which steps are shown below in Figure \ref{fig:preprocessor_pipeline}.

\begin{figure}[H]
\centering
\begin{tikzpicture}[node distance=0.45cm, auto]
\node (story_in) [] {Story};
\node (tokenize) [block, right =of story_in] {Tokenize};
\node (simplify) [block, right =of tokenize] {Simplify};
\node (prune) [block, right =of simplify] {Prune};
\node (homogenise) [block, right =of prune] {Homogenise};
\node (story_out) [block, right =of homogenise] {Preprocessed Story};
\draw [->] (story_in) -- (tokenize);
\draw [->] (tokenize) -- (simplify);
\draw [->] (simplify) -- (prune);
\draw [->] (prune) -- (homogenise);
\draw [->] (homogenise) -- (story_out);
\end{tikzpicture}
\caption{Preprocessor Steps}
\label{fig:preprocessor_pipeline}
\end{figure}

\section{Tokenization and Simplification}

With the help of \textbf{CoreNLP}, we can assign a POS tag to each word, or \textit{token}, from the input story. Using this information, we can now make a number of simplifications which will make the sentence structure more consistent throughout.

\subsection{Punctuation}

\section{Sentence Pruning and Homogenisation}

\section{Example}

\textcolor{red}{\textbf{\hl{TODO}}}

\section{Expandability}

\textcolor{red}{\textbf{\hl{TODO}}}