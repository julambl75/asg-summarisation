\section{Overview}

In order to... as shown in Figure \ref{fig:preprocessor_pipeline}.

\begin{figure}[H]
\centering
\begin{tikzpicture}[node distance=0.55cm, auto]
\node (story_in) [] {Story};
\node (tokenize) [block, right =of story_in] {Tokenize};
\node (simplify) [block, right =of tokenize] {Simplify};
\node (prune) [block, right =of simplify] {Prune};
\node (homogenise) [block, right =of prune] {Homogenise};
\node (story_out) [block, right =of homogenise] {Preprocessed Story};
\draw [->] (story_in) -- (tokenize);
\draw [->] (tokenize) -- (simplify);
\draw [->] (simplify) -- (prune);
\draw [->] (prune) -- (homogenise);
\draw [->] (homogenise) -- (story_out);
\end{tikzpicture}
\caption{Preprocessor Steps}
\label{fig:preprocessor_pipeline}
\end{figure}

\section{Tokenization and Simplification}

\section{Sentence Pruning and Homogenisation}

\section{Expandability}

\textcolor{red}{\textbf{\hl{TODO}}}