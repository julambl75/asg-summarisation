\section{Architecture Overview}

The central part of the pipeline, which performs story summarization, revolves around the ASG task of \textsc{SumASG}. To build on top of this foundation, we first have the \textsc{Preprocessor}, as well as some final post-processing and scoring mechanisms. We will call this augmented task \textsc{SumASG*}, and describe each step in the following chapters. A diagram of the entire pipeline can be seen in Figure \ref{fig:main_pipeline}.

\begin{figure}[H]
\centering
\begin{tikzpicture}[node distance=0.3cm, auto]
\node (story) [] {Story};
\node (preprocessor) [block, right =of story] {Preprocessor};
\node (asg) [block, right =of preprocessor] {SumASG};
\node (score) [block, right =of asg] {Post-Processing/Scoring};
\node (summaries) [right =of score] {Scored Summaries};
\draw [->] (story) -- (preprocessor);
\draw [->] (preprocessor) -- (asg);
\draw [->] (asg) -- (score);
\draw [->] (score) -- (summaries);
\end{tikzpicture}
\caption{Main Pipeline}
\label{fig:main_pipeline}
\end{figure}

\section{Initial Motivation}

Given plenty of time and a large amount of training data in a specific format, a neural network is able to closely approximate the definition of a summary.

In contrast, using logic means that we can hard-code this definition directly into our program. By carefully constructing its structure, we can get results with just a short list of rules. Because it is a logic program, we know that it will always produce a complete and valid output, as long as the program is correct.

\section{Example}

Throughout this paper, we shall be using the example of the story of Peter Little to illustrate the different steps of our pipeline, as shown below in Figure \ref{fig:peter_little}.

\begin{figure}[H]\
\begin{subfigure}{\textwidth}
\begin{displayquote}
There was a curious little boy named Peter Little. He was interested in stars and planets. So he was serious in school and always did his homework. When he was older he studied physics and maths. He studied hard for his exams and became an astrophysicist. Now he is famous.
\end{displayquote}
\caption{Story of Peter Little}
\vspace{\baselineskip}
\end{subfigure}
\begin{subfigure}{\textwidth}
\begin{displayquote}
\textbf{A.} Peter Little was interested in space so he studied hard and became a famous astrophysicist.\\
\textbf{B.} Peter Little was curious about astronomy. He was serious in school. Now he is famous.\\
\caption{Reference summaries}
\end{displayquote}
\end{subfigure}
\caption{Example of the task of summarization for the story of Peter Little}
\label{fig:peter_little}
\end{figure}